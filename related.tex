\noindent
\textbf{Related work.}
The first interactive Proof of Proof-of-Work scheme was presented
in~\cite{popow}. The Non-Interactive Proofs of Proof-of-Work (NIPoPoWs)
primitive was introduced in~\cite{nipopow}, in which a superblock-based
construction was also introduced. Their superblock NIPoPoWs are only proven
secure in the fixed difficulty model. Follow up works have improved their
efficiency~\cite{superblocks} and applied them for cross-chain information
transfers~\cite{pow-sidechains}. Additional improvements in which both
NIPoPoW succinctness and security are achieved simultaneously are presented
in~\cite{logspace}.
FlyClient~\cite{flyclient} gives an alternative probabilistic construction for a
NIPoPoW which leverages the Merkle Mountain Range data structure to create a
longer interlink vector which can give rise to more efficient proofs.
Through a variation of their protocol which uses Merkle Segment Trees, they
provide a construction which might work in variable difficulty settings, but
fall short of providing a proof which takes into account the full model and
prove results only in expectation and not with high probability.
Summa~\cite{summa} put forth an alternative implementation for a NIPoPoW, but
their scheme is susceptible to an attack~\cite{summa-composability} and is not
proven secure in the cryptographic model.
